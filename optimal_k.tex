Xác định số lượng cụm tối ưu trong một tập dữ liệu là vấn đề cơ bản trong phân cụm Kmeans, yêu cầu người dùng chỉ định số lượng cụm k được tạo. Ý tưởng đằng sau Kmeans bao gồm xác định các cụm k sao cho tổng biến thể trong cụm là tối thiểu. Đây được xem là một nhược điểm của thuật toán này. Phần dưới đây trình bày một vài phương pháp giúp xác định số cụm k hợp lý nhất.\par
\subsection{Thuật toán Elbow}
Tư tưởng chính của phương pháp phân cụm phân hoạch (như k-means) là định nghĩa 1 cụm sao cho tổng biến thiên bình phương khoảng cách trong cụm là nhỏ nhất, tham số này là WSS (Within-cluster Sum of Square). Elbow method chọn số sụm k sao cho khi thêm vào  một cụm khác thì không làm cho WSS thay đổi nhiều.